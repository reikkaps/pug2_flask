\documentclass{beamer}

\usepackage[utf8]{inputenc}
\usepackage{listings}

\title[Flask] % (optional, only for long titles)
{Flask Microframework}
\subtitle{für Eilige}
\author[Kaps] % (optional, for multiple authors)
{Reik(o) Kaps}

\date[2016] % (optional)
{Python User Group Hannover 2016}
\subject{Python}

\begin{document}
\frame{\titlepage}
  \begin{frame}
    \frametitle{Flask für Eilige}
    \textbf{\url{http://flask.pocoo.org/}}
    \begin{itemize}
    \item Entwicklungsserver und Debugger
    \item Unit-Test-Unterstützung
    \item Jinja2 Templates
    \item Secure Cookies
    \item 100\% WSGI 1.0 kompatibel
    \item Unicode
    \item gut dokumentiert
    \end{itemize}
  \end{frame}

  \begin{frame}
    \frametitle{Installation}
    \begin{itemize}
    \item 
      \textbf{Python 2.7 oder besser 3.x; virtualenv} \\
      \emph{apt-get install python3 virtualenv}
    \item
      Virtuelle Umgebung anlegen ... \\
      \emph{virtualenv -p /usr/bin/python3 venv}
    \item
      Ins Virtualenv wechseln ... \\
      \emph{. venv/bin/activate}
    \item
      Flask innerhalb des Venv installieren ... \\
      \emph{pip install Flask}

    \item \textbf{Fertig!}
    \end{itemize}

    %More content goes here
  \end{frame}

  \begin{frame}
    \frametitle{Sehr einfache Flask-App}

    \lstinputlisting[language=Python]{simple.py}
  \end{frame}

  \begin{frame}
    \frametitle{Vielen Dank}
    \framesubtitle{für die Aufmerksamkeit!}
    \includegraphics[scale=0.5]{flask_logo.png} \\
    \url{https://github.com/reikkaps/pug2_flask}
  \end{frame}
    
% etc
\end{document}
